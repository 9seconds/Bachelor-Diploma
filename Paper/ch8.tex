\chapter*{Заключение}
\addcontentsline{toc}{chapter}{Заключение}

Для линейных систем с непрерывным временем и шумами, зависящими от состояния и управления, получены необходимые и достаточные условия стабилизации, дающие параметрическое описание (параметризацию) всех линейных стабилизирующих управлений со статической обратной связью по выходу, которые обеспечивают экспоненциальную устойчивость в среднем квадратическом замкнутой системы. На основе этих результатов получены достаточные условия, с помощью которых нахождение матрицы усиления стабилизирующего управления сводится к задаче оптимизации при ограничениях в виде матричных линейных уравнений и неравенств.

Особенностью являлось то, что изначально поставленная задача была невыпукла, в следствие чего не имела удобных методов решения, основанных на достаточно проработанном математическом аппарате.\br

Были получены как алгоритмы робастной стабилизации, так и одновременной. Получен итерационный алгоритм, дающий удобный способ нахождения матриц усиления, с оговоркой на ограниченную применимость. Полученные результаты были успешно применены к решению задачи синтеза пассифицирующего управления с обратной связью по выходу, дан алгоритм решения.\br

Исследования проводились с помощью свободного программного обеспечения SciLab, синтаксического анализатора SciYalmip, являющегося свободной же реализацией известного пакета YALMIP, а также решателя CSDP\footnote{{\fontfamily{cmtt}\selectfont http://infohost.nmt.edu/$\sim$borchers/csdp.html}} под управлением операционной системы GNU/Linux. Успешность решения поставленных задач подтвердила достаточно высокое качество продуктов, поддерживаемых сообществом, сравнимое, а иногда и опережающее проприетарные аналоги.
