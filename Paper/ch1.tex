\chapter*{Введение}
\addcontentsline{toc}{chapter}{Введение}

В настоящее время, проблема стабилизации обратной связи по выходу является одной из ключевых в современной теории управления. К сожалению, полностью она еще не решена, существуют лишь отдельные методы решения частных же задач. Проблема стабилизации обратной связи по выходу намного сложнее стабилизации по состоянию. Для последней проблемы уже существует хорошо разработанный математический аппарат\cite{BOYD} решения линейных матричных неравенств и уравнений, позволяющий получить решение либо точно, либо с достаточной точностью.

Широкое распространение получил \emph{линейно-квадратичный регулятор}\footnote{В англоязычной литературе принята аббревеатура \emph{LQR} (от \emph{Linear quadratic regulator\/})}~--- вид оптимального регулятора, использующего квадратичный функционал качества. Это задача, в которой динамическая система описывается линейными дифференциальными уравнениями, а показатель качества представляет собой квадратичный функционал, называется \emph{задачей линейно-квадратичного управления\/}. Общая постановка задачи такова: пусть у нас есть система дифференциальных уравнений и критерий оптимальности, описанные следующим образом

$$
\dot{x}=Ax+Bu{,}~~~~~J=\int_0^\infty \left( x^TQx + u^TRu \right)\,dt \longrightarrow \min\mbox{.}
$$

Известно\cite{BOYD}, что закон управления по отрицательной обратной связи должен иметь вид $u=-R^{-1}B^TPx$, где матрица $P$ находится из алгебраического уравнения Риккати:

$$
A^TP + PA - PBR^{-1}B^TP + Q = 0\mbox{.}
$$

Также широкое распространение получила задача \emph{линейно-квадратичного гауссовского управления}\footnote{Аналогично, в англоязычной литературые приняты аббревеатура \emph{LQG} и ее расшифровка~--- \emph{Linear quadratic Gaussian control}}~--- набор методов и математического аппарата теории управления для синтеза систем управления с отрицательной обратной связью для линейных систем с аддитивным гауссовским шумом. Синтез проводится путём минимизации заданного квадратичного функционала.

\begin{displaymath}
\left\{ \begin{array}{lcr}
         \dot{x}(t)& = & Ax(t) + Bu(t) + \xi(t)\mbox{,} \\
         y(t)& = &Cx(t) + Du(t) + \eta(t)\mbox{.}
        \end{array} \right.
\end{displaymath}
$$
J = \lim\limits_{T \to \infty} E \int\limits_0^T \left( x(t)^TRx(t) + u^T(t)Qu(t) \right)\,dt \longrightarrow \min\mbox{.}
$$

Где $\xi(t)$~--- возмущения, действующие на объект управления; $\eta(t)$~--- погрешность измерения. Матрицы $R$ и $Q$ представляют собой параметры функционала и являются положительно-определенными.\br

Основной целью работы является исследование задачи LQR-парамет\-ризации обратной связи по выходу для систем с непрерывным временем и марковскими переключениями. Основываясь на результатах исследования, предложены различные способы, методы и алгоритмы решения задачи. Полученные результаты могут быть непосредственно использованы при проектировании систем с помощью различных программных пакетов, как то MATLAB (с решателем YALMIP\footnote{ {\fontfamily{cmtt}\selectfont http://control.ee.ethz.ch/$\sim$joloef/wiki/pmwiki.php}. Автор пакета: Löfberg J. }), или с открытым пакетом SciLab (используя аналогичную наработку, SciYalmip\footnote{{\fontfamily{cmtt}\selectfont http://www.laas.fr/OLOCEP/SciYalmip/} Авторы пакета: Пакшин П., Соловьев С.}).\br

LQR-параметризация используется для синтеза систем управления. Проблема управляемости играет важную роль в адаптивных системах и заключается в следующем: должно существовать такое обратное управление и линейная комбинация вектора выхода $y$, вектора сигналов $\omega$ и соотвествующего ему вектора наблюдений $z$, такие, что замкнутая система будет также управляема относительно $z$. В современной теории управления достаточно хорошо изучен случай, когда $z=y$: в этом случае говорят о полной наблюдаемости. Однако, это все же достаточно жесткое ограничение, так как под него попадают лишь те системы, в которых число входов и выходов совпадает. Следующим шагом развития стало изучение случая, когда $z=Gy$ для заданной матрицы $G$. Очевидно, что это снимает ограничение на количество входов и выходов, однако вводит новую сложность~--- определение (если заранее неизвестна) $G$. Сравнительно хорошо изучен случай, когда $z=Gy+D\omega$~--- случай так называемого \emph{робастного адаптивного управления}.

Поиск такой матрицы $G$~--- задача весьма нетривиальная, представляющая сама собой достаточно сложный объект исследования. Решение, как правило, разбивают на 2 этапа: построение робастной стабилизирующей матицы усиления $F$; построение матрицы $G$ на базе полученной $F$. Сами по себе, эти задачи могут быть эффективно решены с помощью определенных матричных неравенств и уравнений. Однако условия, обеспечиваемые этими уравнениями, к сожалению, почти всегда являются достаточными, но никак не необходимыми. Следовательно, подходы не универсальны, применимы лишь для узкого класса задач. Впрочем, это никоим образом не снижает их высокой практической ценности.\br

Работа устроена следующим образом: в разделе 1 приведенная задача будет четко формализована. В разделе 2 будут предложены некоторые способы решения задачи, основанные на теоретических сведениях. В разделе 3 задача будет обобщена на линейные системы с марковскими переключениями в непрерывном времени, будут даны алгоритмы одновременной и робастной стабилизаций, основанные на теореме о матрице усиления. В разделе 4 полученные результаты будут перенесены на проблему наблюдаемости выхода. В разделе 5 будет показано практическое применение полученных результатов.
