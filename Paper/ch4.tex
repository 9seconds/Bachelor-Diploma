\chapter[Параметризация в стохастических системах]{Параметризация стабилизирующих управлений в стохастических системах}

В этом разделе будет рассмотрено расширение задачи \vref{eq:1/1} на случай стохастических систем. Рассматривается частный случай, при котором система имеет марковские переключения режимов работы. Предыдущие методы~--- $W$- и $P$-проблемы~--- распространяются и на такую модификацию с незначительными изменениями, однако ввиду сужения области исследования, становится возможным использовать и несколько иные подходы, представляющие из себя законченные алгоритмы. Подробнее они будут исследованы в разделе 4. Целью этого же раздела является формализация расширенной задачи и некоторые ценные теоретические сведения, являющиеся основой для дальнейшей работы.\br


% ------------------------------------------------------------------------------------------------------


\section{Формализация задачи}

Задачу \vref{eq:1/1} на случай стохастических систем можно расширить следующим образом. Положим, что система с марковскими переключениями описывается следующей системой уравнений:

\begin{equation}
\label{eq:3/1}
\left\{ \begin{array}{l}
\dot{x}(t) = A\left(r(t)\right)x(t) + B\left(r(t)\right)u(t)\mbox{,} \\
y(t) = C\left(r(t)\right)x(t)\mbox{.}
\end{array} \right.
\end{equation}

Здесь $x(t) \in \Rv{n}$~-- вектор состояний динамической системы; $y(t) \in \Rv{k}$~--- вектор выхода динамической системы; $u(t) \in \Rv{m}$~--- вектор управлений. Матрицы $A$, $B$ и $C$ имеют соответствующие размерности. $r(t)$~($t~\geqslant~0$)~--- процесс переключений режимов работы, принимающий значения из конечного множества $\mathbb{N} = \left\{1,2, \ldots, N\right\}$. Этот процесс моделируется однородной марковской цепью с вероятностями переходов

\begin{eqnarray}
\label{eq:3/2}
\mathbf{P}\big(r(t+h) = j~|~r(t) = i \big) = \left\{ \begin{array}{lr}
\pi_{ij} + o(h) & \mbox{при } i \not= j\mbox{,} \\
1 + \pi_{ij} + o(h) & \mbox{при } i=j\mbox{;}
\end{array} \right. \\
i,j = 1,2,\ldots,N\mbox{.} \nonumber
\end{eqnarray}

Вероятность $\pi_{ij}>0$~--- вероятность переключения из режима $i$ в момент времени $t$ в режим $j$ в момент времени $t+h$ (полагаем $h>0$). Из определения \vref{eq:3/2} и свойств вероятностей следует, что $\pi_{ii} = -\sum_{i\not=j}^N\pi_{ij}$ ($i,j=1,2,\ldots,N$). То обстоятельство, что $\pi_{ii}<0$ не является существенным, так как является чисто конструктивной особенностью и может быть легко исправлено\cite{CDC} (что приведет к усложнению математических выкладок, но лишь по части их громоздкости).

Аналог матрицы вероятностей переходов в марковской цепи~--- матрица $\Pi \eqdef \|\pi_{ij}\|_{\left\{i,j=1,2,\ldots,N\right\}}$.\br

Закон управления с обратной связью определяется совершенно аналогично:

\begin{equation}
\label{eq:3/4}
u(t)=F_iy(t) \mbox{~при } r(t)=i\mbox{.}
\end{equation}

Задача поставлена.



% ------------------------------------------------------------------------------------------------------



\section{Теорема о существовании матрицы усиления}

Будет рассмотрена и доказана теорема, которая дает основные теоретические результаты, на базе которых будут строиться методы решения задачи \vref{eq:3/1} в следующем разделе. Эта теорема основывается на LQR-концепциях и расширяет уже имеющиеся сведения\cite{GLXKA}, касаемые класса систем с марковскими переключениями.

\begin{teo}
\label{teo:3/1}
Матрица усиления $F_i$, обеспечивающая экспоненциальную устойчивость систем \ref{eq:3/1}-\ref{eq:3/4} в среднеквадратическом смысле, существует тогда и только тогда, когда найдутся такие матрицы параметров $Q_i\colon~Q_i = Q_i^T \geqslant 0$ и $R_i\colon~R_i = R_i^T \geqslant 0$, что будет выполнено следующее матричное уравнение:

\begin{equation}
\label{eq:3/5}
F_iC = R_i^{-1}[B_i^TH_i + L_i]\mbox{.}
\end{equation}

Матрицы $H_i\colon~H_i = H_i^T > 0$ и $L_i$ ($L_i \in \mathbb{N}$) есть решения следующей системы матричных неравенств:

\begin{eqnarray}
\label{eq:3/6}
\left\{ \begin{array}{l}
A_i^TH_i + H_iA_i - H_iB_iR_i^{-1}B_i^TH_i + Q_i \\
+ \sum\limits_{i=1}^N\pi_{ij}H_j + L_i^TR_i^{-1}L_i < 0\mbox{,}
\end{array} \right. \\
i \in \mathbb{N}\mbox{.} \nonumber
\end{eqnarray}

\end{teo}

\emph{Необходимость}.
Пусть $F_i$~--- искомая матрица усиления. Тогда будет существовать\cite{KM} положительно определенное решение $H_i\colon~H_i = H_i^T$ следующей системы неравенств:

$$
A_{ci}^TH_i + H_iA_{ci} + \sum\limits_{i=1}^N\pi_{ij}H_j + Q_i + (F_iC_i)^TR_iF_iC_i < 0\mbox{,}~~~~i \in \mathbb{N}\mbox{,}
$$

где $A_{ci} \eqdef A_i - B_iF_iC_i$, $Q_i = Q_i \geqslant 0$ и $R_i = R_i^T \geqslant 0$. После перегруппировки слагаемых, получаем, что

\begin{eqnarray}
\label{eq:3/7}
A_i^TH_i + H_iA_i + (F_iC_i)^TR_iC_iF_i - (F_iC_i)^TB_iH_i - \nonumber \\
H_iB_iF_iC_i + \sum\limits_{i=1}^N\pi_{ij}H_j + Q_i < 0\mbox{,}~~~~i \in \mathbb{N}\mbox{.}
\end{eqnarray}

Положим

\begin{equation}
\label{eq:3/8}
K_i \eqdef F_iC_i - R_i^{-1}B^T_iH_i\mbox{.}
\end{equation}

Перепишем \vref{eq:3/7} в следующей форме:

\begin{eqnarray}
\label{eq:3/9}
A_i^TH_i + H_iA_i - H_iB_iR_i^{-1}B_i^TH_i + \nonumber \\
K_i^TR_iK_i + \sum\limits_{i=1}^N\pi_{ij}H_j + Q_i < 0\mbox{.}
\end{eqnarray}

Затем положим

\begin{equation}
\label{eq:3/10}
L_i \eqdef R_iK_i\mbox{.}
\end{equation}

Подставляя \vref{eq:3/10} в \vref{eq:3/9} и учитывая, что $FC$~--- константная матрица, непосредственно получаем \vref{eq:3/5} и \vref{eq:3/6}.\br

\emph{Достаточность}.
Пускай существуют матрицы $H_i\colon~H_i = H_i^T$ и $F_i$, удовлетворяющие \vref{eq:3/5} и \vref{eq:3/6}. Определим матрицы $K_i$ и $L_i$ такими же, как  в \vref{eq:3/8} и \vref{eq:3/10} соотвественно. Тогда справедливо, что

\begin{eqnarray}
\label{eq:3/11}
0 > A_i^TH_i + H_iA_i - H_iB_iR_i^{-1}B_i^TP_i + \nonumber \\
\sum\limits_{i=1}^N\pi_{ij}H_j + L_i^TR_iL_i + Q_i = \nonumber \\
(A_i - B_iF_iC_i)^TH_i + H_i(A_i - B_iF_iC_i) + \nonumber \\
\sum\limits_{i=1}^N\pi_{ij}H_j + (F_iC_i)^TR_iF_iC_i + Q_i\mbox{,}~~~i \in \mathbb{N}\mbox{.}
\end{eqnarray}

Откуда следует справедливость \vref{eq:3/7}, следовательно $F_i$~--- искомая матрица усиления.\qed\br
