\chapter{Методы решения задачи LQR-параметризации}

Теорема \vref{teo:3/1} дает основной теоретический результат, лежащий в основе всех методов решения задачи LQR-параметризации в линейных системах с непрерывным временем и марковскими переключениями. Следует, однако, отметить тот факт, что почти все они носят достаточный характер, не являясь необходимыми. Поэтому если вдруг один подход не срабатывает, то это вовсе не означает, что задача не решаемая, следует попробовать их все. И даже в том случае, если ни один из них не сработает, определенно утверждать неразрешимость задачи все равно нельзя.

Проблема решения усугубляется еще и тем, что \ref{eq:3/5} и \vref{eq:3/7} не являются стандартными выпуклыми матричными неравенствами.



% --------------------------------------------------------------------------------------------------------------



\section{Метод достаточных выпуклых условий}

Так как \ref{eq:3/5} и \vref{eq:3/7} являются нестандартными матричными неравенствами, которые, к тому же, не выпуклы, то обычные методы и алгоритмы решения не подходят. Возникает идея построения серии вспомогательных выпуклых задач, зная решение которых, требуемую матрицу усиления $F_i$ все же можно построить.

Положим

\begin{eqnarray*}
X_i \eqdef H_i^{-1}{,~~~~} Y_i \eqdef L_iX_i\mbox{,} \\
\Lambda_{11i} \eqdef X_iA_i^T + A_iX_i + \pi_{ii}X_i - B_iR_i^{-1}B_i\mbox{,} \\
\Lambda_{12i} \eqdef X_i\sqrt{Q_i}\sqrt{\pi_{i1}}X_i \cdots \sqrt{\pi_{i\,i-1}}X_i\sqrt{\pi_{i\,i+1}}X_i \cdots \sqrt{\pi_{iN}}X_iY_i^T\mbox{,} \\
\Lambda_{22i} \eqdef \diag[-E~-X_1 \ldots -X_{i-1}~-X_{i+1} \ldots -X_N~-R_i]\mbox{.}
\end{eqnarray*}

В этом случае можно переписать \ref{eq:3/5} и \vref{eq:3/7} в виде матричного неравенства

\begin{equation}
\label{eq:4/1}
\left[ \begin{array}{lr}
    \Lambda_{11i}   & \Lambda_{12i} \\
    \Lambda^T_{12i} & \Lambda_{22i}
\end{array} \right] > 0\mbox{,}~~~i \in \mathbb{N}\mbox{.}
\end{equation}

\begin{equation}
\label{eq:4/2}
F_iC_iX_i = R_i^{-1}(B^T_i + Y_i)\mbox{,}~~~i \in \mathbb{N}\mbox{.}
\end{equation}

Из определения $P$-проблемы \ref{eq:2/3} и теоремы \vref{teo:2/2} следует, что существуют такие матрицы $M_i$ ($i \in \mathbb{N}$), что

\begin{equation}
\label{eq:4/3}
C_iX_i = M_iC_i\mbox{,}~~~i \in \mathbb{N}\mbox{.}
\end{equation}

Тогда \vref{eq:4/2} разрешимо относительно $F_i$ тогда и только тогда\cite{SIG}, когда

\begin{equation}
\label{eq:4/4}
(B_i^T + Y_i)(E - C_i^+\footnote{Под $C^+_i$ понимается псевдообращение матрицы $C_i$ по Муру-Пенроузу.}C_i) = 0\mbox{,}~~~i \in \mathbb{N}\mbox{.}
\end{equation}

Все решения \vref{eq:4/2} непосредственно записываются в виде

\begin{equation}
\label{eq:4/5}
F_i = [R_i^{-1}(B_i^T+Y_i)C^+_i+Z_i(E+C_i^+C_i)]M_i^{-1}\mbox{,}~~~i \in \mathbb{N}\mbox{,}
\end{equation}

где $Z_i$ ($i \in \mathbb(N)$)~--- произвольная матрица подходящей размерности.\br

Следовательно, для некоторых матриц $Q_i\colon~Q_i = Q_i^T \geqslant 0$, $R_i\colon~R_i = R_i^T > 0$ система совокупных линейных уравнений и неравенств \ref{eq:4/1}-\vref{eq:4/3} относительно переменных $X_i$, $Y_i$ и $M_i$ ($i \in \mathbb{N}$) разрешима. Закон управления \vref{eq:3/4} с матрицами усиления, заданными \vref{eq:4/5} будет обеспечивать экспоненциальную устойчивость в среднеквадратическом смысле системы \vref{eq:3/1}.

Можно предложить алгоритм, основанный на приведенных выше рассуждениях.

\begin{alg}
\label{alg:4/1} Метод достаточных выпуклых условий.
\begin{enumerate}
\item Зададим матрицы $Q_i$ и $R_i$, основываясь на LQR-условиях и разрешим задачи \ref{eq:4/1} и \ref{eq:4/3} относительно переменных $X_i$, $Y_i$ и $M_i$ ($i \in \mathbb{N}$);
\item Если задача на шаге 1 разрешима, то вычисляем матрицу усиления $F_i$ по формуле \vref{eq:4/5} для произвольной $Z_i$ ($i \in \mathbb{N}$).
\end{enumerate}

\end{alg}



% --------------------------------------------------------------------------------------------------------------



\section{Метод решения задачи одновременной стабилизации}

Задача одновременной стабилизации получается как частный случай \ref{eq:3/1}-\vref{eq:3/4} при $F_i \equiv F$ и $\pi_{ij}=0$ ($i,j \in \mathbb{N}$). Тогда система \vref{eq:3/1} перепишется в виде

\begin{eqnarray}
\label{eq:4/6}
\left\{ \begin{array}{l}
\dot{x}(t) = A_ix(t) + B_iu(t)\mbox{,} \\
y(t) = Cx(t)\mbox{;}
\end{array} \right. \\
i \in \mathbb{N}\mbox{,} \nonumber
\end{eqnarray}

а обратная связь как

\begin{equation}
\label{eq:4/7}
u(t) = -Fy(t)\mbox{.}
\end{equation}

Матрица усиления $F$ системы \ref{eq:4/6}-\vref{eq:4/7}, которая обеспечивает экспоненциальную устойчивость в среднеквадратическом смысле, существует тогда и только тогда, когда найдутся матрицы $Q_i\colon~Q_i = Q_i^T \geqslant 0$, $R_i\colon~R_i = R_i^T > 0$, что справедливо уравнение

\begin{equation}
\label{eq:4/8}
FC = R_i^{-1}[B_i^TH_i + L_i]\mbox{,}~~~i \in \mathbb{N}\mbox{.}
\end{equation}

Очевидно, что это уравнение суть преобразованное \ref{eq:3/5} для частного случая. Здесь $H_i\colon~H_i = H_i^T > 0$, а $L_i$ ($i \in \mathbb{N}$)~--- решение системы матричных неравенств

\begin{equation}
\label{eq:4/9}
A_i^TH_i + H_iA_i - H_iB_iR_i^{-1}B_i^TH_i + Q_i + L_i^TR_i^{-1}L_i < 0\mbox{,}~~~ i \in \mathbb{N}\mbox{.}
\end{equation}

Тогда для вычисления матрицы усиления можно предложить алгоритм, сходный решению $\mathcal{H}_\infty$-проблемы с помощью теоремы \vref{teo:2/3}.

\begin{alg}
\label{alg:4/2}
Метод решения задачи одновременной стабилизации.

\begin{enumerate}
\item Положим матрицы $Q_i\colon~Q_i = Q_i^T \geqslant 0$ и $R_i\colon~R_i = R_i^T > 0$, основываясь на LQR-соображениях, и решим следующие матричные неравенства и уравнения относительно переменных $X_i$, $Y_i$, $M_i$ ($i \in \mathbb{N}$) и $K$:
\begin{eqnarray}
\label{eq:4/10}
\left\{ \begin{array}{l}
         \left( \begin{array}{ccc}
                 X_iA_i^T + A_iX_i - B_i^TR_i^{-1}B_i  &  X_i\sqrt{Q_i}  &  Y_i^T \\
                 \sqrt{Q_i}X_i   &   -E   &   0 \\
                 Y_i   &   0   &   -R_i
                \end{array}
         \right) < 0\mbox{,} \\
         CX_i = M_iC\mbox{,} \\
         K \eqdef R_i^{-1}(B_i^T + Y_i)\mbox{,} \\
         (B_i^T + Y_i)(E - C_i^+C_i) = 0\mbox{;}
        \end{array} \right. \\
i \in \mathbb{N}\mbox{.} \nonumber
\end{eqnarray}

\item Если \vref{eq:4/10} разрешима, то вычислим матрицу усиления обратной связи $F$ по следующей формуле:

\begin{equation}
\label{eq:4/11}
F = [R_i^{-1}(B_i^T + Y_i)C_i^+ + Z(E + C_i^+C_i)]M_i^{-1}\mbox{.}
\end{equation}

для произвольных $i \in \mathbb{N}$ и матриц параметров $Z$.

\end{enumerate}

\end{alg}




% --------------------------------------------------------------------------------------------------------------



\section{Метод решения задачи робастной стабилизации}

Рассмотрим еще более частный случай задачи \vref{eq:4/6}. Примем, что пара матриц $(A_i,B_i)$ предствляют собой грани политопа, а $H_i = H_i^T \equiv H > 0$. Эта задача является проблемой робастой устойчивости с политопической неопределенностью

\begin{eqnarray}
\label{eq:4/12}
\left\{ \begin{array}{l}
\dot{x} = \sum\limits_{i=1}^N \xi_i(t)[A_ix(t) + B_iu(t)]\mbox{,} \\
y(t) = Cx(t)
\end{array} \right. \\
i \in \mathbb{N}\mbox{,} \nonumber
\end{eqnarray}

где $\xi_i(t) \geqslant 0$ и $\sum_{i=1}^N\xi_i(t)=1$. Обратная связь, как и в \vref{eq:4/7}, представляется в виде $u(t) = Fy(t)$.\br

Утверждается, что матрица усиления обратной связи $F$ будет обеспечивать экспоненциальную устойчивость системы \vref{eq:4/12}-\ref{eq:4/7} в среднеквадратическом смысле тогда и только тогда, когда найдутся матрицы параметров $Q_i\colon~Q_i = Q_i^T \geqslant 0$ и $R_i\colon~R_i = R_i^T > 0$, такие, что выполняется уравнение, аналогичное \vref{eq:4/8}:

\begin{equation}
\label{eq:4/13}
FC = R_i^{-1}[B_i^TH + L_i]\mbox{,}~~~i \in \mathbb{N}\mbox{.}
\end{equation}

где $L_i$~--- решение соответствующей системы матричных неравенств:

\begin{equation}
\label{eq:4/14}
A_i^TH + HA_i - HB_iR_i^{-1}B_i^TH + Q_i + L_i^TR_i^{-1}L_i < 0\mbox{,}~~~ i \in \mathbb{N}\mbox{.}
\end{equation}

Основываясь на этих результатах, предлагается алгоритм, схожий с алгоритмом \vref{alg:4/2}.

\begin{alg}
\label{alg:4/3}
Метод решения задачи робастной стабилизации.

\begin{enumerate}
\item Положим матрицы $Q_i\colon~Q_i = Q_i^T \geqslant 0$ и $R_i\colon~R_i = R_i^T > 0$, основываясь на LQR-соображениях, и решим следующие матричные неравенства и уравнения относительно переменных $X_i$, $Y_i$, $M_i$ ($i \in \mathbb{N}$) и $K$:
\begin{eqnarray}
\label{eq:4/15}
\left\{ \begin{array}{l}
         \left( \begin{array}{ccc}
                 X_iA_i^T + A_iX_i - B_i^TR_i^{-1}B_i  &  X_i\sqrt{Q_i}  &  Y_i^T \\
                 \sqrt{Q_i}X_i   &   -E   &   0 \\
                 Y_i   &   0   &   -R_i
                \end{array}
         \right) < 0\mbox{,} \\
         CX_i = M_iC\mbox{,} \\
         K \eqdef R_i^{-1}(B_i^T + Y_i)\mbox{,} \\
         (B_i^T + Y_i)(E - C^+C) = 0\mbox{;}
        \end{array} \right. \\
i \in \mathbb{N}\mbox{.} \nonumber
\end{eqnarray}

\item Если \vref{eq:4/15} разрешима, то вычислим матрицу усиления обратной связи $F$ по следующей формуле:

\begin{equation}
\label{eq:4/16}
F = [R_i^{-1}(B_i^T + Y_i)C^+ + Z(E + C^+C)]M_i^{-1}\mbox{.}
\end{equation}

для произвольных $i \in \mathbb{N}$ и матриц параметров $Z$.

\end{enumerate}

\end{alg}



% --------------------------------------------------------------------------------------------------------------


\pagebreak
\section{Итерационный алгоритм}

Из теоремы \vref{teo:3/1} непосредственно следует итерационный алгоритм, основной идеей которого является последовательное вычисление последовательности матриц $P_n$, которое должно сходится к требуемой матрице.

\begin{alg}
\label{alg:4/4}

Итерационный алгоритм.

\begin{enumerate}

\item
Задаем некоторые произвольные $Q\colon~Q=Q^T>0$ и $R\colon~R=R^T>0$. Полагаем $n=0$, $L_0=0$. Решаем систему билинейных матричных неравенств относительно $P=P^T>0$ и $K$.

\begin{eqnarray}
\label{eq:4/17}
\left\{ \begin{array}{l}
(A-BK)^TP(A-BK) - P + \sum\limits_{i=1}^N \gamma_j^2(A_j-B_jK)^TP(A_j-B_jK) \\
~~~~~~~~+ Q + K^TRK < 0\mbox{;} \nonumber
\end{array} \right. \\
j \in \{1,2,\ldots,N\}\mbox{,} \nonumber
\end{eqnarray}

Эти неравенства достаточно просто сводятся к линейным с помощью замен $X \eqdef P^{-1}$ и $Y \eqdef KX$. Находим начальное приближение матрицы стабилизирующих управлений как $K_0 = YX^{-1}$. Находим начальные приближения матриц замкнутой системы $\bar{A}_0 = A-BK_0$ и $\bar{A}_{i0} = A_i - B_iK_0$ ($i \in \{1,2,\ldots,N\}$).


\item
$n$-ая итерация.\br

Вычисляем $P_n$ как решение линейных матричных неравенств

\begin{eqnarray}
\label{eq:4/18}
\left\{ \begin{array}{l}
\bar{A}_n^TP\bar{A}_n - P_n + \sum\limits_{j=1}^N \gamma_j^2\bar{A}_{jn}^TP_n\bar{A}_{jn} + Q + K_nRK_n < 0\mbox{;}
\end{array} \right. \\
P_n > 0\mbox{.} \nonumber
\end{eqnarray}

Изменяем $K$ следующим образом:

\begin{equation}
\label{eq:4/19}
K_{n+1} = [B^TP_nB + \Gamma(P) + R]^{-1}[B^TP_nA + \Theta(P_n) + L_n][I - V_2V_2^T]\mbox{.}
\end{equation}

Изменяем $L$ следующим образом:

\begin{equation}
\label{eq:4/20}
L_{n+1} =[B^TP_nB + \Gamma(P_n) + R]K_n - [B^TP_nA + \Theta(P_n)]\mbox{.}
\end{equation}

Матрицы замкнутой системы изменяются соотвественно: $\bar{A}_{n+1}=A - BK_{n+1}$, $\bar{A}_{in+1} = A_i - B_iK_{n+1}$ ($i \in \{1,2,\ldots,N\}$).


\item
Проверка сходимости.\br

Если $\|P_{n+1}-P_n\| / \|P_n\| < \varepsilon$, то переходим к шагу 4. В противном случае, увеличиваем $n$ на единицу и идем к шагу 2.

Здесь $\varepsilon$~--- заданная точность, малое число; $\|~\|$~--- некоторая подчиненная матричная норма.



\item
Окончание.\br

Полагаем $P \eqdef K_{n+1}$ и вычисляем $F$ по формуле

\begin{equation}
\label{eq:4/21}
F = [B^TPB + \Gamma(P) + R]^{-1}[B^TPA + \Theta(P) + L]C^+\mbox{.}
\end{equation}

\end{enumerate}
\end{alg}

Очевидным недостатком итерационного алгоритмы является отсутствие условий сходимости итерационного процесса. В частности, процесс может зациклится (например, в случае, когда $P_{n+1} = -P_n$) или вовсе сойтись к иному решению (критерий остановки по точности, очевидно, вытекает из сходимости к нужному решению, но не наоборот). Впрочем, метод вполне может успешно быть использован на практике, так как вышеприведенные ситуации имеют место достаточно редко.
