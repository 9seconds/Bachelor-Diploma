\chapter{Постановка основной задачи}

Пусть наша система с непрерывным временем задана следующей системой уравнений:

\begin{equation}
\label{eq:1/1}
\left\{ \begin{array}{rcl}
        \dot{x}&=&Ax + Bu\mbox{,} \\
        y&=&Cx\mbox{.}
       \end{array} \right.
\end{equation}

где $x=x(t)$, $x \in \Rv{n}$; $y=y(t)$, $y \in \Rv{n_y}$; $u=u(t)$, $u \in \Rv{n_u}$; $t \in [0, +\infty)$). Матрицы таковы, что $A \in \Rs{n}{n}$, $B \in \Rs{n_u}{n}$, $C \in \Rs{n_y}{n}$.

В 1890 году Ляпунов А.\,М. доказал, что система \vref{eq:1/1} стабилизируема тогда и только тогда, когда существуют положительно определенная матрица $P>0$\footnote{Здесь и далее выражение $M>0$ ($M<0$) означает положительную (отрицателную) определенность матрицы $M$} и $F$ подходящих размерностей такие, что выполняется неравенство:

\begin{equation}
\label{eq:1/2}
P(A-BF) + (A-BF)^TP < 0\mbox{.}
\end{equation}

Домножим \vref{eq:1/2} с обеих сторон на матрицу $W \eqdef P^{-1}$. Получаем:

\begin{equation}
\label{eq:1/3}
(A-BF)W + W(A-BF)^T < 0\mbox{.}
\end{equation}

Сделаем замену $L \eqdef FW$:

\begin{equation}
\label{eq:1/4}
AW + WA^T - LB - L^TB^T < 0\mbox{.}
\end{equation}

Получили известную задачу\cite{BOYD}, разрешимую в переменных $W$ и $L$ тогда и только тогда, когда пара матриц $A$ и $B$ будут устойчивыми. В этом случае управление определяется как $u=-LW^{-1}$. Проблема определения разрешимости может быть легко проделана засчет известных алгоритмов\cite{BOYD}. Например, можно использовать эллиминацию матричных переменных. Для этого нужно \vref{eq:1/4} переписать в виде

\begin{equation}
\label{eq:1/5}
AW + WA^T - \sigma BB^T < 0\mbox{.}
\end{equation}

Доказано, что $\sigma$ существует. Эквивалентой формой \vref{eq:1/5} является

$$
\hat{B}^T(AW+WA^T)\hat{B} < 0\mbox{,}
$$

\begin{flushleft}
где $\hat{B}$~--- ортогональное дополнение $B$. В этом случае оптимальное управление $F$ будет записано в виде
\end{flushleft}

\begin{equation}
\label{eq:1/6}
F = -\frac{\sigma}{2}B^TW^{-1}
\end{equation}

\begin{flushleft}
для любой $W>0$. Без ограничения общности можно полагать, что $\sigma=1$. В противном случае, всегда можно привести эквивалентные преобразования, в которых $\sigma$ <<исчезнет>>.\br
\end{flushleft}

Теперь предположим, что желаемая структура управления может быть записана в виде $u=-F_0y$ или, эквивалентно, $u=-Fx$. Очевидно, что $F \eqdef F_0C$. Тогда из \vref{eq:1/3} получаем:

\begin{equation}
\label{eq:1/7}
(A-BF_0C)W + W(A-BF_0C)^T < 0\mbox{,}~~~~~~W>0\mbox{.}
\end{equation}

Теоретическое решение задачи получено, однако оно не применимо на практике: это связано с тем, что неравенство относительно $W$ и $F_0$ не является в общем случае выпуклым, следовательно эффективные методы решения здесь не применимы. Впрочем, с использованием нескольких вспомогательных задач, являющихся выпуклыми, можно проблему в значительной мере упростить.
