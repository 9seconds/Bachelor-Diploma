\chapter{Некоторые методы решения поставленных задач}

\section{$W$-проблема}

$W$-проблема тесно связана с \vref{eq:1/7} и формулируется следующим образом.

Пусть заданы некоторые матрицы $A$, $B$, $C$ (полагаем, что $C$ имеет полный строчный ранг). $W$-проблема заключается в нахождении матриц $W$, $N$ и $M$ таких, что

\begin{equation}
\label{eq:2/1}
\left\{ \begin{array}{l}
         AW + WA^T - BNC - C^TN^TB^T < 0\mbox{,} \\
         W > 0\mbox{,} \\
         MC = CW\mbox{.}
        \end{array}
\right.
\end{equation}

$W$-проблема вызывает двойной интерес. Во-первых, она выпукла, и, следовательно, может быть разрешена эффективными алгоритмами\cite{BOYD}, такими как эллиминация матричных переменных или $\mathcal{S}$-процедура (в частных случаях). Во-вторых, если она разрешима, то будет разрешимой и задача \vref{eq:1/7}.

Покажем это, доказав следующую теорему.

\begin{teo}
\label{teo:2/1}
Если $W$, $N$, $M$ являются решениями \vref{eq:2/1}, то управления

$$
u = -NM^{-1}y = -F_0y
$$

\begin{flushleft}
будут обеспечивать устойчивость системы \vref{eq:1/1}.
\end{flushleft}
\end{teo}

\emph{Доказательство}.
Так как $C$ имеет полный строчный ранг и выполняется условие $MC=CW$, то матрица $M$ также будет иметь полный строчный ранг. Следовательно, $C$ можно представить через уравнение

\begin{equation}
\label{eq:2/2}
C = M^{-1}CW\mbox{.}
\end{equation}

\begin{flushleft}
Примем $F_0 \eqdef -NM^{-1}$. Тогда подставляя \vref{eq:2/2} в \vref{eq:1/7}, после серии очевидных преобразований получим систему \ref{eq:2/1}.\qed
\end{flushleft}


% ----------------------------------------------------------------------


\section{$P$-проблема}

$P$-проблема практически идентична $W$-проблеме, так как получается из нее простой заменой переменных.

Пусть заданы матрицы $A$, $B$, $C$ подходящих размерностей. $B$ имеет полный строковый ранг. $P$-проблема заключается в нахождении, если это возможно, матриц $P$, $N$, $M$ таких, что

\begin{equation}
\label{eq:2/3}
\left\{ \begin{array}{l}
         PA + A^TP - C^TN^TB^T - BNC < 0\mbox{,} \\
         P > 0\mbox{,} \\
         BM = PB\mbox{.}
        \end{array}
\right.
\end{equation}

Совершенно аналогично доказывается теорема о $P$-проблеме.

\begin{teo}
\label{teo:2/2}
Если $P$, $N$, $M$ являются решениями $P$-проблемы \vref{eq:2/3}, то управления

$$
u = -N^{-1}My = -F_0y
$$

\begin{flushleft}
будут обеспечивать устойчивость системы \vref{eq:1/1}.
\end{flushleft}
\end{teo}

\emph{Замечание}. $P$- и $W$-проблемы схожи, так как представляют из себя разный взгляд на одни и те же уравнения. Если основой для $W$-проблемы служит уравнение \vref{eq:1/3}, то для $W$-проблемы~--- \vref{eq:1/2}.\br

Когда $C=E$\footnote{Здесь и далее, под $E$ понимается классическая единичная матрица подходящей размерности.} $W$-проблема значительно упрощается и сводится к стандартным техникам решения матричных неравенств вследствие того, что уравнение $MC=CW$ превращается в обычное равенство $M=W$. В этом случае $P$-проблема предоставляет альтернативные способы вычисления стабилизирующих матриц усиления обратной связи, которые могут представлять практический интерес в определенном классе задач\cite{CRUSIUS}.


% --------------------------------------------------------------------------------------------------



\section{Проблема $\mathcal{H}_\infty$}

$\mathcal{H}_\infty$-управление~--- метод теории управления для синтеза оптимальных регуляторов. Он является оптимизационным, имеющим дело со строгим математическим описанием предполагаемого поведения замкнутой системы и её устойчивости.

$\mathcal{H}$ является нормой в пространстве Харди\footnote{Расширение $\mathcal{L}^p$-пространства на комплексную плоскость.}. <<Бесконечность>> говорит о выполнении минимаксных условий в частотной области. $\mathcal{H}_\infty$~--- норма динамической системы, имеющая смысл максимального усиления системы по энергии.\br

Оказывается, что $W$- и $P$-проблемы могут быть использованы для решения $\mathcal{H}_\infty$-проблемы. Более того, результаты имеют настолько общий характер, что могут быть использованы и на $\mathcal{H}_2$ без потери выразительности.

Положим, что система описывается как

\begin{equation}
\label{eq:2/4}
\left\{ \begin{array}{l}
         \dot{x} = Ax + B_uu + B_\omega\omega\mbox{,} \\
         y = C_yx + D_{uy}u + D_{\omega y}\omega\mbox{,} \\
         z = C_zx + D_{uz}u + D_{\omega z}\omega\mbox{.}
        \end{array}
\right.
\end{equation}

\begin{flushleft}
Где $x$, $y$, $z$~--- вектора соответствующих размерностей. $A$, $B_u$, $B_\omega$, $C_y$, $D_{uy}$, $D_{\omega y}$, $C_z$, $D_{uz}$, $D_{\omega z}$~--- матрицы соответсвующих размерностей. Тогда справедлива связывающая теорема.
\end{flushleft}

\begin{teo}
\label{teo:2/3}
Пускай система описана с помощью \vref{eq:2/4}. $D_{uy}$, $D_{\omega y}$~--- нулевые матрицы. Введем обозначение:

$$
\Phi(W,N) \eqdef WA + A^TW - B_uNC_y - C^T_yN^TB^T_u\mbox{.}
$$

Предположим, что выполняется следующее матричное неравенство с ограничениями:

\begin{equation}
\label{eq:2/5}
\left\{
\begin{array}{l}
 \left( \begin{array}{ccc}
         \Phi(W,N)         & B_\omega     & WC^T_\omega - C^T_yN^TD^T_{uz} \\
         B^T_\omega        & -\gamma^2E   & D^T_{\omega z} \\
         C_zW - D_{uz}NC_y & D_{\omega z} & -E
        \end{array}
 \right) < 0\mbox{,} \\
MC_y=C_yW\mbox{,} \\
W > 0\mbox{.}
\end{array}
\right.
\end{equation}

Пусть \vref{eq:2/5} разрешимо относительно матриц $W$, $N$ и $M$. Если управление $u=-NM{-1}$, то для замкнутой относительно $\omega$ и $z$ системы норма $\mathcal{H}_\infty$ будет такой, что $\|G_{\omega z}\|_\infty < \gamma$.

\end{teo}

\emph{Доказательство}.
Известно\cite{BOYD}, что норма $\mathcal{H}_\infty$ будет меньше $\gamma$ тогда и только тогда, когда существует матрица $W>0$ такая, что

\begin{equation}
\label{eq:2/6}
\left( \begin{array}{ccc}
        WA^Tf+A_fW    & B_f           & WC^T_f \\
        B^T_f         & -\gamma^2E    & D^T_f \\
        C_fW          & D_f           & -E
       \end{array}
\right) < 0\mbox{,}
\end{equation}

где $A_f$, $B_f$, $C_f$ и $D_f$ представляют из себя матрицы пространства состояний исходной системы.

Применим к \vref{eq:2/4} управление $u=-F_0y$ и получим, что в этом конкретном случае $A_f = A-B_uF_0C_y$, $B_f = B_\omega$, $C_f = C_z - D_{uz}F_0C_y$ и $D_f = D_\omega$. Очевидно, что \vref{eq:2/6} выполняется при $A_f = A-B_uF_0C_y$ и $F_0 = NM^{-1}$. Совершенно аналогично доказательству теоремы \vref{teo:2/1} можно показать, что выполнение \vref{eq:2/6} влечет за собой и справедливость \vref{eq:2/5}. Следовательно, замкнутая система удовлетворяет требуемому условию $\|G_{\omega z}\|_\infty < \gamma$.\qed\br



% ------------------------------------------------------------------


\section{Анализ достаточности для $W$- и $P$-проблем}

Проблема разрешимости $W$- и $P$-проблем зависит от той интерпретации состояний, которую выбирают для описания динамической системы. Однако существуют преобразования, которые позволяют решить эти задачи, даже если это невозможно в изначальной интерпретации.

Пусть $x$ определяет вектор состояния системы \vref{eq:1/1}. Преобразуем систему следующим образом: $x_0 = T_0^{-1}x$. Определим $T$ как $T \eqdef T_0T_0^T$ и произведем замену матриц ($A$, $B$, $C$, $P$, $W$) на ($T_0^{-1}AT_0$, $T_0^{-1}B$, $CT_0$, $T_0^TPT_0$, $T_0W{T_0^{-1}}^T$). Тогда \vref{eq:2/1} и \vref{eq:2/3} перепишутся следующим образом:

\begin{equation}
\label{eq:2/7}
\left\{ \begin{array}{l}
AW + WA^T - BNCT - TC^TN^TB^T < 0\mbox{,} \\
MCT = CW\mbox{,} \\
W > 0\mbox{,} \\
T > 0\mbox{;}
\end{array} \right.
\end{equation}

\begin{equation}
\label{eq:2/8}
\left\{ \begin{array}{l}
PA + A^TP - C^TN^TB^TT^{-1} - T^{-1}BNC < 0\mbox{,} \\
T^{-1}BM = PB\mbox{,} \\
P > 0\mbox{,} \\
T^{-1} > 0\mbox{.}
\end{array} \right.
\end{equation}

Совершенно очевидно, что после подобных преобразований $W$- и $P$-проблемы могут стать разрешимыми, даже если в исходных пространствах они таковыми не являлись. Но, к сожалению, \vref{eq:2/7} и \vref{eq:2/8} не являются выпуклыми относительно ($W$, $N$, $M$, $T$) и ($P$, $N$, $M$, $T^{-1}$) соотвественно. Задача определения преобразования, после которого исходные задачи станут разрешимыми до сих пор является открытой.

Впрочем, можно отметить тот немаловажный факт, что $W$-проблема инвариантна относительно преобразований $T_0$, удовлетворяющих уравнению $CT=RC$ для некоторой матрицы $R$ и $T\colon~T=T_0T_0^T$: это следует из \vref{eq:2/7}, где можно провести обычную замену переменных $N$ и $M$, сводящих уравнение к аналогичному. Аналогичный факт устанавливается и для $P$-проблемы: она будет инвариантна относительно преобразования $T_0$, которое удовлетворяет уравнению $T^{-1}B=BR$. Очевидным примером такого преобразования является $T \equiv E$ (ортогональные преобразования $T_0T^T_0=E$).

Хорошо известен\cite{PERES} следующий факт.

\begin{teo}
\label{teo:2/4}
Система \vref{eq:1/1} стабилизируема обратной связью по состоянию тогда и только тогда, когда существует подобное преобразование, что $C=(I~~~0)$ и \vref{eq:1/4} разрешима с $W$ и $L$, удовлетворяющим следующим структурным ограничениям:

\begin{eqnarray*}
W&=&\left( \begin{array}{cc}
            W_1 &   0 \\
            0   & W_2
           \end{array} \right)\mbox{,} \\
L&=&\left( L_1~~~0 \right)\mbox{.}
\end{eqnarray*}

\end{teo}

Отметим, что $C=(I~~~0)$ и ограничение $MC=CW$ в \vref{eq:2/1} эквивалентны тому, что $W$ имеет такую структуру. Если взять $L=(N~~~0)$, то можно показать\cite{PERES}, что система \vref{eq:1/1} стабилизируема через обратную связь по состоянию тогда и только тогда, когда существует преобразование подобия, превращающее $W$-проблему в разрешимую.\br

Аналогичные рассуждения справедливы и при рассмотрении $P$-проб\-лемы.
